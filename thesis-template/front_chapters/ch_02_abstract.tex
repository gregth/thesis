\abstract
\addcontentsline{toc}{chapter}{\abstractname}
\phantomsection
% APPROVED

Kubernetes is the de facto container orchestrator choice for every company going
cloud-native. It can efficiently orchestrate a large number of containers via a
powerful declarative management interface, reducing operational burdens for the
cluster admins. 

The Kubernetes storage interface allows the integration of different storage
systems, which can get used as persistent volumes by the workload. The use of
local persistent volumes over remote persistent storage offers the benefit of
performance: local disks offer higher IOPS and throughput and lower latency
compared to remote storage systems.

Currently, the Cluster Autoscaler does not support autoscaling with local
storage. Moreover, the Scheduler does not consider the available capacity of
local storage when scheduling Pods.

Enabling the seamless operation of the Cluster Autoscaler and Scheduler on
Kubernetes clusters that use local storage systems is crucial for enterprises to
reduce costs. The local storage will enable their disk-intensive workload to
complete faster, and the Scheduler will ensure that the workload units run on
nodes that have the requested storage capacity. At the same time, the Cluster
Autoscaler will scale the cluster at an appropriate size for the workload to run
without any excess resource waste.

In this diploma thesis, we propose and implement extensions for the Cluster
Autoscaler and the Scheduler to seamlessly operate with local storage. During
this thesis, we deployed the proposed extended Cluster Autoscaler and Scheduler
to enterprises, and they used it successfully at large production clusters.
Moreover, we started contributing parts of the proposed design upstream.

\subsection*{Keywords}
Kubernetes, Cluster Autoscaler, Scheduler, Local Storage, Logical Volume
Manager, Capacity Tracking

