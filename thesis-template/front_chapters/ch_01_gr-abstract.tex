\grabstract
\addcontentsline{toc}{chapter}{\grabstractname}
\phantomsection
% APPROVED
\vspace{-0.5cm}
Ο Kubernetes είναι η de facto επιλογή ενορχηστρωτή containers για κάθε εταιρεία
που χρησιμοποιεί cloud native εφαρμογές, καθώς δύναται να
ενορχηστρώνει αποδοτικά ένα μεγάλο αριθμό containers μέσω μιας ισχυρής
δηλωτικής διεπαφής διαχείρισης, μειώνοντας έτσι τις λειτουργικές επιβαρύνσεις
για τους διαχειριστές των συστοιχιών.

Η διεπαφή αποθήκευσης του Kubernetes επιτρέπει την ενσωμάτωση διαφορετικών
αποθηκευτικών συστημάτων, τα οποία εν συνεχεία μπορούν να χρησιμοποιηθούν ως
μόνιμοι τόμοι από το φορτίο εργασίας.  Η χρήση τοπικών μόνιμων τόμων έναντι
απομακρυσμένου μόνιμου αποθηκευτικού χώρου προσφέρει το πλεονέκτημα των υψηλών
επιδόσεων: οι τοπικοί δίσκοι προσφέρουν υψηλότερο αριθμό IOPS, μεγαλύτερους
ρυθμούς μετάδοσης και χαμηλότερη καθυστέρηση σε σύγκριση με τα απομακρυσμένα
συστήματα αποθήκευσης.

Επί του παρόντος, ο Cluster Autoscaler δεν υποστηρίζει την αυτόματη κλιμάκωση σε
συστοιχίες με τοπικούς μόνιμους τόμους, ενώ ο Scheduler δεν λαμβάνει υπόψη του
την ελεύθερη χωρητικότητα στους τοπικούς δίσκους των κόμβων κατά τη
χρονοδρομολόγηση των Pods.

Η απρόσκοπτη λειτουργία του Cluster Autoscaler και του Scheduler σε συστοιχίες
\en{Kubernetes} με τοπική αποθήκευση είναι πολύ σημαντική για τις εταιρείες,
καθώς τους επιτρέπει να έχουν συστοιχίες, που αξιοποιούν τα πλεονεκτήματα της
τοπικής αποθήκευσης, ενώ προσαρμόζεται δυναμικά το μέγεθός τους και το φορτίο
εκτελείται αδιάλειπτα.

Στην παρούσα διπλωματική εργασία προτείνουμε και υλοποιούμε επεκτάσεις για τον
Cluster Autoscaler και τον Scheduler, προκειμένου να λειτουργούν απρόσκοπτα με
τον τοπικό αποθηκευτικό χώρο. Κατά τη διάρκεια αυτής, λοιπόν, εγκαταστήσαμε με
επιτυχία τις επεκτάσεις του Cluster Autoscaler και Scheduler στις συστοιχίες
διαφόρων εταιρειών. Επιπλέον, ξεκινήσαμε να συνεισφέρουμε τμήματα του
προτεινόμενου σχεδιασμού upstream.
% TODO: Reread last paragraph

\subsection*{Λέξεις-Κλειδιά}
Kubernetes, Cluster Autoscaler, Scheduler, Χρονοδρομολόγηση, Τοπική Αποθήκευση,
Συστοιχία
