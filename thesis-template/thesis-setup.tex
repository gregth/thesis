%
% Setup code & template fields
%

% Font selection
\defaultfontfeatures{Scale=MatchLowercase,Mapping=tex-text}
\setmainfont{MinionPro}[
   Path=./static/fonts/MinionPro/ ,
   UprightFont =*-Regular ,
   BoldFont=*-Semibold ,
   ItalicFont=*-Italic ,
   BoldItalicFont=*-SemiboldItalic,
   Extension = .otf
 ]
\setsansfont{MyriadPro}[
   Path=./static/fonts/MyriadPro/ ,
   UprightFont =*-Regular ,
   BoldFont=*-Bold ,
   ItalicFont=*-Italic ,
   BoldItalicFont=*-BoldItalic,
   Extension = .otf
 ]
\setmonofont{Consolas}[
   Path=./static/fonts/Consolas/ ,
   UprightFont=*-Regular ,
   BoldFont=*-Bold ,
   ItalicFont=*-Italic ,
   BoldItalicFont=*-BoldItalic,
   Extension = .ttf
 ]

% No headers on empty pages before new chapter
\makeatletter
\def\cleardoublepage{\clearpage\if@twoside \ifodd\c@page\else
   \hbox{}
   \thispagestyle{plain}
   \newpage
   \if@twocolumn\hbox{}\newpage\fi\fi\fi}
\makeatother \clearpage{\pagestyle{plain}\cleardoublepage}

% English intro
\newcommand\introname{Introduction}
\newcommand\intro{%
    \chapter*{\introname
        \markboth{%
           \MakeUppercase\introname}{\MakeUppercase\introname}}%
}

% Greek intro
\newcommand\grintroname{Εισαγωγή}
\newcommand\grintro{%
    \chapter*{\grintroname
        \markboth{%
           \MakeUppercase\grintroname}{\MakeUppercase\grintroname}}%
}

% English abstract
\newcommand\abstractname{Abstract}
\newcommand\abstract{%
    \cleardoublepage
    \section*{\abstractname
        \markboth{}{}}%
}

% Greek abstract
\newcommand\grabstractname{Περίληψη}
\newcommand\grabstract{%
    \cleardoublepage
    % Intentionally set to sectio in order to save space
    \section*{\grabstractname%
        % Don't show any heading
        \markboth{}{}}%
}

% Greek foreword
\newcommand\grforewordname{Αντί Προλόγου}
\newcommand\grforeword{%
    \chapter*{\grforewordname
        \markboth{%
           \MakeUppercase\grforewordname}{\MakeUppercase\grforewordname}}%
}

% Template fields
\ntuaimage{./styles/pyrforos.pdf}

% Greek
\title{Επεκτάσεις για Χρονοδρομολόγηση \\ και Αυτόματη Κλιμάκωση σε Συστοιχίες Kubernetes \\ με Τοπική Αποθήκευση Δεδομένων}
\author{Γρηγόριος Π. Θανάσουλας}
\date{Ιούλιος 2022}

\ntuadoctype{ΔΙΠΛΩΜΑΤΙΚΗ ΕΡΓΑΣΙΑ}

\ntualarge{Ε{\large ΘΝΙΚΟ} Μ{\large ΕΤΣΟΒΙΟ} Π{\large ΟΛΥΤΕΧΝΕΙΟ}}
\ntuasmall{Ε{\normalsize ΘΝΙΚΟ} Μ{\normalsize ΕΤΣΟΒΙΟ} Π{\normalsize ΟΛΥΤΕΧΝΕΙΟ}}
\schoollarge{Σ{\normalsize ΧΟΛΗ} Η{\normalsize ΛΕΚΤΡΟΛΟΓΩΝ} Μ{\normalsize ΗΧΑΝΙΚΩΝ}\\ Κ{\normalsize ΑΙ} Μ{\normalsize ΗΧΑΝΙΚΩΝ} Υ{\normalsize ΠΟΛΟΓΙΣΤΩΝ}}
\schoolsmall{Σ{\small ΧΟΛΗ} Η{\small ΛΕΚΤΡΟΛΟΓΩΝ} Μ{\small ΗΧΑΝΙΚΩΝ} Κ{\small ΑΙ} Μ{\small ΗΧΑΝΙΚΩΝ} Υ{\small ΠΟΛΟΓΙΣΤΩΝ}}
\deptlarge{Τ{\normalsize ΟΜΕΑΣ} Τ{\normalsize ΕΧΝΟΛΟΓΙΑΣ} Π{\normalsize ΛΗΡΟΦΟΡΙΚΗΣ} Κ{\normalsize ΑΙ} Υ{\normalsize ΠΟΛΟΓΙΣΤΩΝ}}
\deptsmall{Τ{\small ΟΜΕΑΣ} Τ{\small ΕΧΝΟΛΟΓΙΑΣ} Π{\small ΛΗΡΟΦΟΡΙΚΗΣ} Κ{\small ΑΙ} Υ{\small ΠΟΛΟΓΙΣΤΩΝ}}

\authornominative{Γρηγόριος Π. Θανάσουλας}
\authortitle{Ηλεκτρολόγος Μηχανικός και Μηχανικός Υπολογιστών ΕΜΠ}
\authortitlepost{Ηλεκτρολόγος Μηχανικός και Μηχανικός Υπολογιστών ΕΜΠ}
\advisors{Νεκτάριος Κοζύρης\\ Καθηγητής ΕΜΠ}
\deptsmall{Τ{\small ΟΜΕΑΣ} Τ{\small ΕΧΝΟΛΟΓΙΑΣ} Π{\small ΛΗΡΟΦΟΡΙΚΗΣ} Κ{\small ΑΙ} Υ{\small ΠΟΛΟΓΙΣΤΩΝ}}

\committeeone{Νεκτάριος Κοζύρης\\ Καθηγητής ΕΜΠ}
\committeetwo{Γεώργιος Γκούμας\\ Αν. Καθηγητής ΕΜΠ}
\committeethree{Διονύσιος Πνευματικάτος\\ Καθηγητής ΕΜΠ}
\submitdate{Εγκρίθηκε από την τριμελή εξεταστική επιτροπή την 14η Ιουλίου 2022.}

\place{Αθήνα}
\year{2022}

% English
\entitle{Extensions for Scheduling and Autoscaling\\ on Kubernetes Clusters with Local Storage Considerations}
\enauthor{Grigorios P. Thanasoulas}
\endate{July 2022}

\enntuadoctype{DIPLOMA THESIS}

\enntualarge{N{\large ATIONAL} T{\large ECHINCAL} U{\large NIVERSITY OF} A{\large THENS}}
\enntuasmall{N{\normalsize ATIONAL} T{\normalsize ECHINCAL} U{\normalsize NIVERSITY OF} A{\normalsize THENS}}
\enschoollarge{S{\normalsize CHOOL OF} E{\normalsize LECTRICAL AND} C{\normalsize OMPUTER} E{\normalsize NGINEERING}}
\enschoolsmall{S{\small CHOOL OF} E{\small LECTRICAL AND} C{\small OMPUTER} E{\small NGINEERING}}
\endeptlarge{D{\normalsize IVISION OF} C{\normalsize OMPUTER} S{\normalsize CIENCE}}
\endeptsmall{D{\small IVISION OF} C{\small OMPUTER} S{\small CIENCE}}

\enauthornominative{Grigorios P. Thanasoulas}
\enauthortitle{Electrical and Computer Engineer NTUA}
\enauthortitlepost{Electrical and Computer Engineer NTUA}
\enadvisors{Nectarios Koziris\\ Professor NTUA}
\encommitteeone{Nectarios Koziris\\ Professor NTUA}
\encommitteetwo{Georgios Goumas\\ Associate Professor NTUA}
\encommitteethree{Dionysios Pnevmatikatos\\ Professor NTUA}
\ensubmitdate{Approved by the three-member examination committee on the 14th of July 2022.}
\enplace{Athens}
\enyear{2022}

% pandoc
\providecommand{\tightlist}{%
  \setlength{\itemsep}{0pt}\setlength{\parskip}{0pt}}

%include PDF files
\usepackage{pdfpages}

% flaot figures
\usepackage{float}

\usepackage{multicol} %<<<<<<<<<<<

% algorithms
\usepackage[]{listings}
\usepackage[linesnumbered]{algorithm2e}
% \SetKwIF{If}{ElseIf}{Else}{if~(\endgraf}{\endgraf)~then}{else if}{else}{end if}%
%\SetKwIF{If}{Else}{if~(\endgraf}{\endgraf)~then}{else}{end if}%

% Modify margin on the left of an enumeration
\usepackage{enumitem}

% Mutliple lines as input 
% https://tex.stackexchange.com/questions/64204/mutliple-inputs-with-line-breaking
\newcommand\myinput[1]{%
  \settowidth\mylen{\KwIn{}}%
  \setlength\hangindent{\mylen}%
  \hspace*{\mylen}#1\\}

\SetEndCharOfAlgoLine{.}

%% Use custom packages
\lstdefinestyle{myCustomMatlabStyle}{
  language=Matlab,
  numbers=left,
  stepnumber=1,
  numbersep=10pt,
  tabsize=4,
  showspaces=false,
  showstringspaces=false
}
\usepackage{amsmath}
\newcommand{\lar}{\(\leftarrow\)}


\usepackage{fancyvrb}
\newcommand{\VerbBar}{|}
\newcommand{\VERB}{\Verb[commandchars=\\\{\}]}
\DefineVerbatimEnvironment{Highlighting}{Verbatim}{commandchars=\\\{\}}
% Add ',fontsize=\small' for more characters per line
\newenvironment{Shaded}{}{}
\newcommand{\AlertTok}[1]{\textcolor[rgb]{1.00,0.00,0.00}{\textbf{#1}}}
\newcommand{\AnnotationTok}[1]{\textcolor[rgb]{0.38,0.63,0.69}{\textbf{\textit{#1}}}}
\newcommand{\AttributeTok}[1]{\textcolor[rgb]{0.49,0.56,0.16}{#1}}
\newcommand{\BaseNTok}[1]{\textcolor[rgb]{0.25,0.63,0.44}{#1}}
\newcommand{\BuiltInTok}[1]{#1}
\newcommand{\CharTok}[1]{\textcolor[rgb]{0.25,0.44,0.63}{#1}}
\newcommand{\CommentTok}[1]{\textcolor[rgb]{0.38,0.63,0.69}{\textit{#1}}}
\newcommand{\CommentVarTok}[1]{\textcolor[rgb]{0.38,0.63,0.69}{\textbf{\textit{#1}}}}
\newcommand{\ConstantTok}[1]{\textcolor[rgb]{0.53,0.00,0.00}{#1}}
\newcommand{\SpecialStringTok}[1]{\textcolor[rgb]{0.73,0.40,0.53}{#1}}
\newcommand{\StringTok}[1]{\textcolor[rgb]{0.25,0.44,0.63}{#1}}
\newcommand{\VariableTok}[1]{\textcolor[rgb]{0.10,0.09,0.49}{#1}}
\newcommand{\VerbatimStringTok}[1]{\textcolor[rgb]{0.25,0.44,0.63}{#1}}
\newcommand{\WarningTok}[1]{\textcolor[rgb]{0.38,0.63,0.69}{\textbf{\textit{#1}}}}
\usepackage{longtable,booktabs}
% Fix footnotes in tables (requires footnote package)
\IfFileExists{footnote.sty}{\usepackage{footnote}\makesavenoteenv{longtable}}{}

\newcommand{\co}[1]{\texttt{#1}}
\newcommand{\tbf}[1]{\textbf{#1}}

%linebreak
  %\newfontfamily\ttfamily[Scale=.7]{Monaco}
  %\usepackage{fontspec}


\lstdefinestyle{mystyle}{
    backgroundcolor=\color{backcolour},   
    commentstyle=\color{codegreen},
    keywordstyle=\color{magenta},
    numberstyle=\tiny\color{codegray},
    stringstyle=\color{codepurple},
    basicstyle=\ttfamily\footnotesize,
    breakatwhitespace=false,         
    breaklines=true,                 
    captionpos=b,                    
    keepspaces=true,                 
    numbers=left,                    
    numbersep=5pt,                  
    showspaces=false,                
    showstringspaces=false,
    showtabs=false,                  
    tabsize=2
}

  \lstset{
	tabsize=4,
  numbers=left,
  stepnumber=1,
	rulecolor=,
        basicstyle=\scriptsize\linespread{0.8}\ttfamily,
        upquote=true,
        %aboveskip={1\baselineskip},
        columns=fixed,
        showstringspaces=false,
        extendedchars=true,
        breaklines=true,
        prebreak = \raisebox{0ex}[0ex][0ex]{\ensuremath{\hookleftarrow}},
        frame=single,
        showtabs=false,
        showspaces=false,
        showstringspaces=false,
        % keywordstyle=\color[rgb]{0,0,1},
        keywordstyle=\color{Blue},
        commentstyle=\color{Gray},
        stringstyle=\color{Mahogany}
}


%% YAML
%% https://www.latex4technics.com/?note=187E
\newcommand\YAMLcolonstyle{\color{Mahogany}\mdseries}
\newcommand\YAMLkeystyle{\color{black}\bfseries}
\newcommand\YAMLvaluestyle{\color{Blue}\mdseries}

\makeatletter

% here is a macro expanding to the name of the language
% (handy if you decide to change it further down the road)
\newcommand\language@yaml{yaml}

\expandafter\expandafter\expandafter\lstdefinelanguage
\expandafter{\language@yaml}
{
  keywords={true,false,null,y,n},
  keywordstyle=\color{darkgray}\bfseries,
  basicstyle=\scriptsize\linespread{0.7}\YAMLkeystyle\ttfamily,                                 % assuming a key comes first
  sensitive=false,
  comment=[l]{\#},
  morecomment=[s]{/*}{*/},
  commentstyle=\color{purple}\ttfamily,
  stringstyle=\YAMLvaluestyle\ttfamily,
  moredelim=[l][\color{orange}]{\&},
  moredelim=[l][\color{magenta}]{*},
  moredelim=**[il][\YAMLcolonstyle{:}\YAMLvaluestyle]{:},   % switch to value style at :
  morestring=[b]',
  morestring=[b]",
  literate =    {---}{{\ProcessThreeDashes}}3
                {>}{{\textcolor{red}\textgreater}}1     
                {|}{{\textcolor{red}\textbar}}1 
                {\ -\ }{{\mdseries\ -\ }}3,
}

% switch to key style at EOL
\lst@AddToHook{EveryLine}{\ifx\lst@language\language@yaml\YAMLkeystyle\fi}
\makeatother

\newcommand\ProcessThreeDashes{\llap{\color{cyan}\mdseries-{-}-}}

% https://github.com/Tedxz/xjtuthesis-x/issues/1
\makeatletter
\newenvironment{breakablealgorithm}
  {% \begin{breakablealgorithm}
   \begin{center}
     \refstepcounter{algorithm}% New algorithm
     \hrule height.8pt depth0pt \kern2pt% \@fs@pre for \@fs@ruled
     \renewcommand{\caption}[2][\relax]{% Make a new \caption
       {\raggedright\textbf{\ALG@name~\thealgorithm} ##2\par}%
       \ifx\relax##1\relax % #1 is \relax
         \addcontentsline{loa}{algorithm}{\protect\numberline{\thealgorithm}##2}%
       \else % #1 is not \relax
         \addcontentsline{loa}{algorithm}{\protect\numberline{\thealgorithm}##1}%
       \fi
       \kern2pt\hrule\kern2pt
     }
  }{% \end{breakablealgorithm}
     \kern2pt\hrule\relax% \@fs@post for \@fs@ruled
   \end{center}
  }
\makeatother


% Full references
% Updated definition, see explanation below
\newcommand*{\fullref}[1]{\hyperref[{#1}]{\autoref*{#1} \nameref*{#1}}} % One single link
