\chapter{\grintroname} 
\label{chapter:gr-introduction}

Σε αυτό το κεφάλαιο περιγράφουμε συνοπτικά το αντικείμενο της εργασίας µας.
Παρουσιάζουμε µία σύντομη επισκόπηση της δουλειάς μας και εντοπίζουμε τις όποιες
αδυναμίες υπάρχουν στις τρέχουσες προσεγγίσεις. Συνεχίζοντας, δίνουμε µία βασική
σύνοψη του μηχανισμού που χτίσαμε. Τέλος, παρουσιάζουμε τη δομή αυτής της
διπλωματικής εργασίας.

\section{Κίνητρο} \label{section:gr-intro_motivation}

Ο Kubernetes είναι η de facto επιλογή ενορχηστρωτή containers για κάθε εταιρεία
που αναπτύσσει cloud native εφαρμογές. Η δημοτικότητα του Kubernetes προκύπτει
από το γεγονός ότι κάνει τη διαχείριση του κύκλου ζωής των εφαρμογών, που
βασίζονται σε containers, πολύ πιο εύκολη για τα DevOps, μέσω ενός δηλωτικού API.
Οι προγραμματιστές μπορούν να χρησιμοποιήσουν αυτό το API, για να περιγράψουν την
επιθυμητή κατάσταση μιας εφαρμογής με όρους Pods, Services, κλ.π., και οι
ελεγκτές του Kubernetes εκτελούν άμεσα ενέργειες, για να φέρουν την
παρατηρούμενη κατάσταση τους συστήματος στην επιθυμητή κατάσταση. Ο Kubernetes
μπορεί να εκτελεστεί τόσο σε φυσικές εγκαταστάσεις, όσο και στους περισσότερους παρόχους cloud
υπηρεσιών, όπως οι Amazon Web Services (AWS), Google Cloud και Microsoft Azure.
Το γεγονός αυτό αποτελεί καθοριστικό παράγοντα για την επιτυχία του.

% Να το ξαναγράψω αυτό;
Με το packaging του κώδικα και των εξαρτήσεων του σε containers, κάθε ομάδα
ανάπτυξης λογισμικού μπορεί να χρησιμοποιήσει τυποποιημένες μονάδες κώδικα για
την εκτέλεση λειτουργιών, ενώ διευκολύνεται η κλιμάκωση των εφαρμογών σε
οποιοδήποτε μέγεθος. Τα containers μπορούν να χρησιμοποιηθούν για το packaging
ολόκληρων εφαρμογών, έτσι ώστε να μπορούν να μεταφερθούν για εκτέλεση στο
υπολογιστικό νέφος, χωρίς να χρειάζεται να υποστεί αλλαγές ο κώδικας. Ο
Kubernetes λειτουργεί ως πλατφόρμα ενορχηστρωτή containers και επιτρέπει σε
μεγάλο αριθμό containers να εκτελείται και να συνεργάζεται αρμονικά,
ελαττώνοντας τις λειτουργικές επιβαρύνσεις για τους διαχειριστές της συστοιχίας.

% TODO: κλπ 
Ο Kubernetes Scheduler και ο Cluster Autoscaler λειτουργούν από κοινού, για να
εξασφαλίσουν ότι το φορτίο εργασίας της συστοιχίας εκτελείται αδιάλειπτα. Ο
Scheduler διασφαλίζει ότι οι μονάδες φορτίου εργασίας, τα Pods,
χρονοδρομολογούνται σε κόμβους της συστοιχίας που έχουν επαρκείς πόρους, όπως
μνήμη, CPU, αποθηκευτικό χώρο, κ.λπ., ενώ ο Cluster Autoscaler διατηρεί τον
κατάλληλο αριθμό κόμβων στη συστοιχία, κλιμακώνοντάς την προς τα πάνω (προσθήκη
κόμβων) ή προς τα κάτω (αφαίρεση κόμβων), επιτρέποντας στις μονάδες φορτίου
εργασίας να εκτελούνται απρόσκοπτα, χωρίς να ξοδεύονται τυχόν πλεονάζοντες
πόροι.  Από το τελευταίο συνεπάγεται ότι ο Cluster Autoscaler είναι ένα
στοιχείο, που επιτρέπει στις εταιρείες να βελτιστοποιήσουν το κόστος της
υπολογιστικής υποδομής, κλιμακώνοντας δυναμικά τον αριθμό των κόμβων στη
συστοιχία με βάση το φορτίο εργασίας, ανταποκρινόμενος δηλαδή στη δυναμική
ζήτηση των πόρων. Χωρίς τον Cluster Autoscaler, οι εταιρείες θα ήταν
υποχρεωμένες να χρησιμοποιούν μία συστοιχία σταθερού πλήθους κόμβων, που θα είχε
σαν συνέπεια είτε την ύπαρξη υποχρησιμοποιούμενων κόμβων στη συστοιχία, με
αποτέλεσμα τη χρέωση των αδρανών κόμβων, είτε το φορτίο εργασίας δεν θα είχε
τους απαραίτητους πόρους για να εκτελεστεί.

Ο συνδυασμός του Kubernetes με τοπικούς μόνιμους τόμους επιτρέπει στους χρήστες
να έχουν πρόσβαση στον τοπικό αποθηκευτικό χώρο ενός κόμβου στη συστοιχία, μέσω
της διεπαφής \texttt{PersistentVolumeClaim} του Kubernetes με έναν απλό και φορητό
τρόπο. Το πρωταρχικό πλεονέκτημα των τοπικών μόνιμων τόμων σε σχέση με την
απομακρυσμένη μόνιμη αποθήκευση (π.χ., network-attached volumes) είναι η
επίδοση: οι τοπικοί δίσκοι προσφέρουν συνήθως υψηλότερα IOPS και
χαμηλότερη καθυστέρηση σε σύγκριση με τα απομακρυσμένα συστήματα αποθήκευσης.
Ενδεικτικά, με τη χρήση δίσκων μη πτητικής μνήμης (NVMe) σε έναν κόμβο, ο
τελικός χρήστης μπορεί να επωφεληθεί από μία τεράστια αύξηση των επιδόσεων κατά
την εκτέλεση εφαρμογών, που απαιτούν μεγάλη χρήση του αποθηκευτικού χώρου. Στην
περίπτωση των εταιρειών, ο τοπικός αποθηκευτικός χώρος τους επιτρέπει
να βελτιστοποιήσουν την ταχύτητα με την οποία εκτελούν εργασίες υψηλής απαίτησης
σε operations δίσκου, όπως πειράματα μηχανικής μάθησης, εργασίες ανάλυσης μεγάλων
δεδομένων κ.λπ, το οποίο αποτελεί κρίσιμο παράγοντα για τα κέρδη τους.

Ωστόσο, αυτή τη στιγμή, η εργασία με τοπικό μόνιμο αποθηκευτικό χώρο έχει ορισμένα προβλήματα,
που δεν έχουν επιλυθεί: ο Scheduler δεν χρονοδρομολογεί το
φορτίο εργασίας λαμβάνοντας υπόψη τον διαθέσιμο τοπικό αποθηκευτικό χώρο και
επίσης, ο Cluster Autoscaler δεν μπορεί να κλιμακώσει προς τα κάτω ή προς τα
πάνω συστοιχίες, όταν γίνεται χρήση τοπικών μόνιμων τόμων. Η απρόσκοπτη
λειτουργία του Cluster Autoscaler σε συστοιχίες Kubernetes, που χρησιμοποιούν
τοπικό αποθηκευτικό χώρο, είναι πολύ σημαντική για τις εταιρείες. Η τοπική
αποθήκευση θα τους επιτρέψει να εκτελούν φορτίο εργασίας απαιτητικό σε
operations δίσκου αποτελεσματικά και γρήγορα, ο Cluster Autoscaler θα διατηρεί
το κατάλληλο μέγεθος της συστοιχίας και ο Autoscaler θα διασφαλίσει ότι κάθε
μονάδα του φορτίου εργασίας εκτελείται σε κατάλληλο είδος κόμβου. Αυτός ο τριπλός
συνδυασμός λειτουργεί εμφανώς προς την κατεύθυνση της μείωσης του κόστους
υποδομής, πράγμα το οποίο είναι επιδίωξη κάθε εταιρείας.

Με κίνητρο τη σημασία και τον αντίκτυπο της χρήσης τοπικής μόνιμης αποθήκευσης
σε συστοιχίες Kubernetes, στην παρούσα διπλωματική εργασία προτείνουμε επεκτάσεις για τον
Scheduler και τον Cluster Autoscaler, ώστε να καταστεί δυνατή η απρόσκοπτη
λειτουργία τους με τοπική αποθήκευση. Πιο συγκεκριμένα, η εργασία μας είναι
διττή:

\begin{itemize}
      \item Προτείνουμε έναν μηχανισμό, έτσι ώστε ο Cluster Autoscaler να μπορεί
            να κλιμακώσει προς τα κάτω και προς τα πάνω συστοιχίες με
            τοπικό μόνιμο αποθηκευτικό χώρο.
      \item Επεκτείνουμε τον Scheduler, για να διασφαλίσουμε ότι το φορτίο
            εργασίας χρονοδρομολογείται σε κόμβους, που διαθέτουν την απαιτούμενη
            ποσότητα τοπικού αποθηκευτικού χώρου.
\end{itemize}

\section{Διατύπωση Προβλήματος} \label{section:gr-intro_problem_statement}

Όπως εξηγήσαμε στην ενότητα \ref{section:gr-intro_motivation}, η παρούσα
διπλωματική εργασία έχει ως κίνητρο τα οφέλη που προσφέρει η τοπική μόνιμη αποθήκευση σε
συνδυασμό με τον Kubernetes. Ωστόσο, η λειτουργία του Kubernetes Scheduler και
του Cluster Autoscaler, με την τοπική αποθήκευση παρουσιάζει ορισμένα
προβλήματα, τα οποία δεν έχουν επιλυθεί ακόμη:
\begin{itemize}
      \tightlist
      \item Ο Scheduler δεν χρονοδρομολογεί το φορτίο εργασίας λαμβάνοντας υπόψη
            τον διαθέσιμο αποθηκευτικό χώρο. Λόγω αυτού, ένα Pod μπορεί να
            κολλήσει χωρίς ποτέ να χρονοδρομολογηθεί σε κατάλληλο κόμβο, αφού ο
            Scheduler αναγκάζεται να επιλέξει στα τυφλά (όσον αφορά στην
            αποθηκευτική χωρητικότητα) κόμβο, στον οποίο, ενδεχομένως, ο τόμος δεν
            μπορεί να διατεθεί, επειδή το υποκείμενο σύστημα αποθήκευσης δεν έχει
            επαρκή χωρητικότητα.
      \item Ο Cluster Autoscaler δεν κλιμακώνει προς τα κάτω συστοιχίες με
            τοπικό μόνιμο αποθηκευτικό χώρο, καθώς θεωρεί πως τα τοπικά δεδομένα
            ζουν σε κάθε κόμβο και η αφαίρεση ενός κόμβου θα οδηγούσε στην
            απώλεια πρόσβασης στα δεδομένα αυτά.
      \item Ο Cluster Autoscaler δεν κλιμακώνει προς τα πάνω συστοιχίες, όταν
            δεν υπάρχει αρκετός τοπικός αποθηκευτικός χώρος για την εκτέλεση του
            φορτίου εργασίας, καθώς δεν διαθέτει μηχανισμό να ενημερωθεί για την
            αποθηκευτική χωρητικότητα ενός κόμβου, που θα προσθέσει.
\end{itemize}

Στην παρούσα διπλωματική εργασία κάνουμε σχεδιαστικές προτάσεις και τις
υλοποιούμε, προκειμένου να ξεπεράσουμε αυτά τα προβλήματα και να καταστεί δυνατή
η απρόσκοπτη λειτουργία του Scheduler και του Cluster Autoscaler με
τοπικό μόνιμο αποθηκευτικό χώρο.

\section{Υπάρχουσες Λύσεις} \label{section:gr-intro_existing_solutions}

Επί του παρόντος, ο Cluster Autoscaler δεν υποστηρίζει αυτόματη κλιμάκωση για
κόμβους που χρησιμοποιούν τοπικό μόνιμο αποθηκευτικό χώρο. Ο Cluster Autoscaler
θεωρεί ότι κάθε τόμος που βασίζεται σε τοπική μόνιμη αποθήκευση είναι
προσβάσιμος μόνο από αυτόν τον κόμβο και, ως εκ τούτου, δεν θα αφαιρέσει τον
κόμβο όπου ζουν τα δεδομένα. Με άλλα λόγια, η κλιμάκωση προς τα κάτω με
τοπικούς τόμους την παρούσα στιγμή είναι αδύνατη. Επιπλέον, δεν υπάρχει
δυνατότητα κλιμάκωσης προς τα πάνω όταν δεν υπάρχει αρκετός τοπικός
αποθηκευτικός χώρος για το φορτίο εργασίας που τον ζητά, καθώς δεν διαθέτει
μηχανισμό να ενημερωθεί για τη χωρητικότητα ενός κόμβου που θα προσθέσει. Την
παρούσα στιγμή δεν υπάρχουν λύσεις για τη διόρθωση αυτών των προβλημάτων.

Όσον αφορά τον Scheduler, η κοινότητα του \en{Kubernetes} έχει υλοποιήσει μια
αρχική υποστήριξη για χρονοδρομολόγηση με τοπικό αποθηκευτικό χώρο. Το
χαρακτηριστικό αυτό ονομάζεται ``\textit{Παρακολούθηση Αποθηκευτικής Χωρητικότητας’’} και
επιτρέπει στους οδηγούς συστημάτων αποθήκευσης να δημοσιεύουν πληροφορίες
σχετικά με την εναπομένουσα χωρητικότητα σε κάθε τμήμα τοπολογίας της
συστοιχίας. Στη συνέχεια, ο Scheduler χρησιμοποιεί αυτές τις πληροφορίες για να
επιλέξει έναν κατάλληλο κόμβο για κάθε Pod που ζητάει την παροχή τόμων. Ωστόσο,
η τρέχουσα προσέγγιση έρχεται με ορισμένους περιορισμούς:

\begin{itemize}
      \item Δεν επιχειρεί να μοντελοποιήσει τον τρόπο με τον οποίο οι αποφάσεις
            χρονοδρομολόγησης επηρεάζουν την αποθήκευση χωρητικότητα. Αυτή είναι
            μια σχεδιαστική απόφαση της ομάδας του Kubernetes με στόχο να
            διευκολύνει την ανάπτυξη του feature, δεδομένου ότι η επίδραση των
            αποφάσεων χρονοδρομολόγησης  στη διαθέσιμη χωρητικότητα μπορεί να
            διαφέρει σημαντικά ανάλογα με τον τρόπο που το σύστημα αποθήκευσης
            χειρίζεται την αποθήκευση. Ως συνέπεια αυτής της απόφασης, με τον
            τρέχοντα σχεδιασμό, ένα Pod που ζητά την παροχή πολλαπλών τόμων
            μπορεί να ανατεθεί σε έναν κόμβο όπου υπάρχει αρκετός χώρος μόνο για
            καθένα από τους τόμους ξεχωριστά, χωρίς να λαμβάνεται υπόψη η
            συνολική ποσότητα του αποθηκευτικού χώρου που απαιτείται για τη
            φιλοξενία όλων των τόμων. Αυτό μπορεί να οδηγήσει στο σενάριο όπου η
            χρονοδρομολόγηση αυτού του Pod μπορεί να αποτύχει μόνιμα: ένας τόμος
            μπορεί να δημιουργηθεί με επιτυχία σε ένα τμήμα τοπολογίας, στο
            οποίο στη συνέχεια δεν θα απομείνει αρκετή χωρητικότητα για τον άλλο
            τόμο. Τότε, κάθε μελλοντική προσπάθεια χρονοδρομολόγησης του Pod θα
            περιορίζεται από τον τόμο που δημιουργήθηκε ήδη. Σε αυτή την
            περίπτωση, είναι απαραίτητη η χειροκίνητη παρέμβαση για την ανάκτηση
            από αυτή την κατάσταση, για παράδειγμα αυξάνοντας τη χωρητικότητα ή
            διαγράφοντας τον τόμο που είχε ήδη δημιουργηθεί.
      \item Η λειτουργία Παρακολούθησης Αποθηκευτικής Χωρητικότητας μπορεί να
            ενεργοποιηθεί μόνο σε συστοιχίες που εκτελούν \ έκδοση Kubernetes
            1.21 ή νεότερη. Αυτή είναι μια απαίτηση που δεν ικανοποιούν όλες οι
            συστοιχίες σε περιβάλλον παραγωγής. Αρκετές εταιρείες χρησιμοποιούν
            προηγούμενες εκδόσεις Kubernetes για λόγους σταθερότητας. Στην
            περίπτωσή μας, οι πελάτες μας χρησιμοποιούν Kubernetes 1.19 και
            1.20, όπου η λειτουργία είναι μη διαθέσιμη.
\end{itemize}

\section{Προτεινόμενη Λύση} \label{section:gr-intro_proposed_solution}

Όπως έχει ήδη εξηγηθεί σε προηγούμενες ενότητες, ο στόχος της παρούσας
διπλωματικής είναι να επιτρέψει την απρόσκοπτη αυτόματη κλιμάκωση και
χρονοδρομολόγηση σε συστοιχίες που αξιοποιούν τοπική μόνιμη αποθήκευση. Για να
το επιτύχουμε αυτό, ο προτεινόμενος σχεδιασμός μας περιλαμβάνει την επέκταση
διαφόρων μερών: του Scheduler, του Cluster Autoscaler και του οδηγού τοπικής
αποθήκευσης Rok CSI της Arrikto.

Συνοπτικά, προτείνουμε τον ακόλουθο σχεδιασμό:
\begin{itemize}
      \item Επεκτείνουμε τον οδηγό τοπικής αποθήκευσης Rok CSI ώστε να αναφέρει
            τη διαθέσιμη χωρητικότητα αποθήκευσης σε κάθε κόμβο της συστοιχίας.
            Το πρόγραμμα οδήγησης θα υπολογίζει τον διαθέσιμο αποθηκευτικό χώρο
            υποβάλλοντας εντολές στον υποκείμενο Local Volume Manager, ενώ θα
            λαμβάνει υπόψη του και τον χώρο που χρειάζεται να δεσμευθεί για την
            αποθήκευση των μεταδεδομένων του τόμου.
      \item Επεκτείνουμε τον Scheduler ώστε να λαμβάνει υπόψη την αναφερόμενη
            διαθέσιμη χωρητικότητα αποθήκευσης κάθε κόμβου κατά την
            χρονοδρομολόγηση ενός Pod που ζητά την παροχή τόμων τοπικού
            αποθηκευτικού χώρου του Rok.
      \item Εγκαθιστούμε στη συστοιχία τον επεκταμένο Scheduler, (τον οποίο
            εφεξής αποκαλούμε ``\textit{Rok Scheduler}'') μαζί με ένα mutating
            webhook, το οποίο θα μεταλλάσσει τα Pods ώστε να αναφέρουν ότι
            πρέπει να χρονοδρομολογηθούν από τον Rok Scheduler.
      \item Αξιοποιούμε τον μηχανισμό του οδηγού αποθήκευσης Rok CSI για τη
            δημιουργία snapshots και για την προστασία των τοπικών τόμων ενός
            κόμβου όταν ένας κόμβος της συστοιχίας πρόκειται να αφαιρεθεί, ώστε
            να μπορούμε να κλιμακώσουμε προς τα κάτω τη συστοιχία.
      \item Επεκτείνουμε τον Cluster Autoscaler ώστε κλιμακώνει προς τα κάτω την
            συστοιχία με τα τοπικά δεδομένα και να συντονίζεται με τον
            μηχανισμό του Rok CSI που τα προστατεύει και λαμβάνει τα αντίγραφα
            ασφαλείας τους.
      \item Επεκτείνουμε τον Cluster Autoscaler ώστε να λαμβάνει υπόψη τη
            χρησιμοποίηση του αποθηκευτικού χώρου όταν αποφασίζει να αφαιρέσει
      έναν κόμβο, καθώς και να ελέγχει αν υπάρχει αρκετός αποθηκευτικός
            χώρος σε άλλους κόμβους για να φιλοξενήσει τα Pods των κόμβων που
            σκοπεύει να αφαιρέσει.
      \item Επεκτείνουμε τον  Cluster Autoscaler ώστε να λαμβάνει υπόψη  τον
            διαθέσιμο αποθηκευτικό χώρο στους κόμβους που προσθέτει κατά το την
            κλιμάκωση προς τα πάνω, δεδομένου ότι μια συστοιχία μπορεί να έχει
            διαμορφωμένες ομάδες κόμβων με διαφορετικό αποθηκευτικό χώρο.
      \item Επεκτείνουμε τον Cluster Autoscaler ώστε να μην αφαιρεί κόμβους μιας
            συστοιχίας που μπορεί να βρίσκονται σε κατάσταση Unready μετά από
            κάποιο σφάλμα, δεδομένου ότι τα τοπικά δεδομένα εξακολουθούν να ζουν
            εκεί και ενδεχομένως απαιτούνται ενέργειες από το διαχειριστή της
            συστοιχίας για την ανάκτηση από αυτή την κατάσταση.
\end{itemize}

\section{Δομή Διπλωματικής Εργασίας} \label{section:gr-intro_outline}

Το ελληνικό τμήμα της διπλωματικής αυτής εργασίας αποτελεί μία σύντομη περίληψη
των αντίστοιχων αγγλικών κεφαλαίων. Ενώ προσπαθούμε να δώσουμε μία επαρκή εικόνα
για το σύνολο της εργασίας μας, κάποιες λεπτομέρειες καθώς και οι πλήρεις
αλγόριθμοι παραλείπονται, και παρατίθενται μόνο στο αγγλικό τμήμα της εργασίας.

Τα επόμενα κεφάλαια του ελληνικού τμήματος της εργασίας, οργανώνονται ως εξής: 

\begin{itemize}
      \item Στο \textbf{Κεφάλαιο 2} παρέχουμε το θεωρητικό υπόβαθρο που είναι
            απαραίτητο για να κατανοήσει ο αναγνώστης την εργασία μας.
      \item Στο \textbf{Κεφάλαιο 3} αναλύουμε τον σχεδιασμό του Kubernetes
            Scheduler και του \en{Cluster Autoscaler}, εκθέτουμε τμήματα του
            μηχανισμού τους και προτείνουμε σχεδιαστικές αλλαγές για να καταστεί
            δυνατή η απρόσκοπτη λειτουργία τους με τοπική μόνιμη αποθήκευση.
      \item Στο \textbf{Κεφάλαιο 4} αναλύουμε την υλοποίηση του σχεδιασμού μας.
      \item Στο \textbf{Κεφάλαιο 5} παρέχουμε μια περίληψη των συνεισφορών μας
            καθώς και πιθανές μελλοντικές κατευθύνσεις.
\end{itemize}