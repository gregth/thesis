\chapter{Υλοποίηση} \label{chapter:gr-implementation}

Σε αυτό το κεφάλαιο περιγράφουμε συνοπτικά την υλοποίηση των προτεινόμενων
αλλαγών σχεδιασμού και τις τεχνολογίες που χρησιμοποιήθηκαν.

\section{Λογισμικό}

Ο προτεινόμενος σχεδιασμός περιλαμβάνει πολλά διαφορετικά μέρη που έπρεπε να επεκταθούν:
\begin{itemize}
    \tightlist
    \item Kubernetes Scheduler, γραμμένο σε Go.
    \item Kubernetes Cluster Autoscaler, γραμμένο σε Go.
    \item Rok CSI driver, γραμμένο σε Python.
\end{itemize}


Εισάγει επίσης ένα νέο στοιχείο , το webhook του scheduler, το οποίο
αναπτύξαμε εξ ολοκλήρου στη γλώσσα Go.

Χτίζουμε τα δομικά μέρη με αναπαράξιμο τρόπο, χρησιμοποιώντας  Docker
containers κατάλληλα για τη γλώσσα που είναι γραμμένο το κάθε μέρος. Για
να περιγράψουμε και να αυτοματοποιήσουμε τη διαδικασία χτισίματος
χρησιμοποιήσαμε Dockerfiles και Makefiles.

Προκειμένου να εγκαταστήσουμε τα δομικά μέρη (Cluster Autoscaler, Rok Scheduler,
Rok Scheduler webhook) στη συστοιχία, γράψαμε YAML manifests που χρησιμοποιούν
το δηλωτικό API του Kubernetes για να περιγράψουν τους απαραίτητα αντικείμενα
που πρέπει να δημιουργηθούν. Για να διευκολύνουμε τη διαχείριση των manifests,
χρησιμοποιήσαμε το εργαλείο Kustomize . Το Kustomize είναι ένα εργαλείο
διαχείρισης manifests που αξιοποιεί τη διαστρωμάτωση για τη διατήρηση των
βασικών ρυθμίσεων των εφαρμογών και των δομικών στοιχείων,  ενώ μπορεί να
επικαλύψει κομμάτια των  YAML manifests (δημιουργώντας patches) που παρακάμπτουν
επιλεκτικά τις προεπιλεγμένες ρυθμίσεις χωρίς να αλλάζουν στην πραγματικότητα τα
αρχικά αρχεία.

Καθώς στο τρέχον κεφάλαιο περιγράφουμε συνοπτικά την υλοποίηση μας, μπορείτε να
δείτε την πλήρη υλοποίηση στο αντίστοιχο αγγλικό κεφάλαιο \ref{chapter:implementation}.