\chapter{Επίλογος} \label{chapter:gr-conclusion}

Σε αυτό το κεφάλαιο, θα επαναδιατυπώσουμε τις συνεισφορές µας και ϑα συνοψίσουμε
τι προσφέρει ο μηχανισμός µας. Τέλος, θα κλείσουμε αυτή τη διπλωματική εργασία
αναφέροντας τι μελλοντικό έργο μπορεί να γίνει ώστε να εμπλουτιστεί ο μηχανισμός
µας και να φτάσει στο σύνολο των δυνατοτήτων του.

\section{Συμπερασματικά Σχόλια}
\label{section:gr-conclusion_concluding_remarks}


Συνολικά, ο πρωταρχικός στόχος αυτής της διπλωματικής ήταν η υλοποίηση ενός
σχεδιασμού που θα επιτρέψει την απρόσκοπτη αυτόματη κλιμάκωση και
χρονοδρομολόγηση σε συστοιχίες με τοπική μόνιμη αποθήκευση. Ο στόχος αυτός
επιτεύχθηκε. Η υλοποίησή μας παραδόθηκε σε αρκετές εταιρείες και εγκαταστάθηκε
στης συστοιχίες παραγωγής τους. Η υλοποίησή μας είναι ιδιαίτερης σημασίας για
τις εταιρείες αυτές, καθώς τους επιτρέπει να επωφεληθούν από όλα τα
πλεονεκτήματα της τοπικής αποθήκευσης, ενώ το μέγεθος της συστοιχίας
προσαρμόζεται δυναμικά στις απαιτήσεις του φορτίου εργασίας για υπολογιστικούς
πόρους και τοπική χωρητικότητα.

Ο σχεδιασμός που υλοποιήσαμε είναι άμεσα συνυφασμένος με το λογισμικό Rok της
Arrikto, καθώς το Rok καθιστά δυνατή τη μετακίνηση των τοπικών τόμων από κόμβο
σε κόμβο χάρη στον αποδοτικό μηχανισμό snapshots που διαθέτει. Παρόλα αυτά, η
λύση μας μπορεί να γενικευτεί και να προσαρμοστεί και σε άλλα συστήματα τοπικής
αποθήκευσης. Ο μακροπρόθεσμος στόχος μας, που εκτείνεται πέρα από το πλαίσιο
αυτής της διπλωματικής, είναι να γενικεύσουμε το σχεδιασμό και να τον
ενσωματώσουμε στο upstream project του Kubernetes. Αυτή είναι μια διαδικασία
στην οποία αρχίσαμε να εμπλεκόμαστε παρακολουθώντας τις συναντήσεις των
Kubernetes Storage
\footnote{https://github.com/kubernetes/community/blob/master/sig-storage/README.md}
και Autoscaling
\footnote{https://github.com/kubernetes/community/blob/master/sig-autoscaling/README.md}
Special Interest Groups και σκοπεύουμε να εμπλακούμε ενεργά προκειμένου να
συνεισφέρουμε το σύνολο του σχεδιασμού upstream. Τη στιγμή που αυτό το κείμενο
γράφεται, έχουν γίνει merge κάποια αρχικά Pull Requests
\footnote{https://github.com/kubernetes/autoscaler/pull/4877}
\footnote{https://github.com/kubernetes/autoscaler/pull/4842} που υποβάλαμε.


\section{Μελλοντικό Έργο} \label{section:gr-conclusion_future_work} Παρόλο που
υλοποιήσαμε αρκετές βελτιώσεις στον Kubernetes Scheduler και το Cluster
Autoscaler, υπάρχουν ακόμη περιθώρια ανάπτυξης. Δεδομένου ότι πρόκειται για μία
επαναληπτική διαδικασία, στις επόμενες επαναλήψεις θα θέλαμε να προσφέρουμε
αυτές τις βελτιώσεις:

\begin{itemize}
      \item Να επεκτείνουμε τον Scheduler ώστε να δεσμεύει τον αποθηκευτικό χώρο
            κατά τη χρονοδρομολόγηση ενός Pod (στο στάδιο \co{Reserve}
            του κύκλου χρονοδρομολόγησης), προκειμένου να αποτρέπονται
            πιθανές συνθήκες ανταγωνισμού.
      \item Να επεκτείνουμε τον Cluster Autoscaler να λαμβάνει υπόψη τον
            αποθηκευτικό χώρο που απαιτείται για τα PVCs πολλαπλών Pods
            κατά την κλιμάκωση προς τα κάτω. Ο τρέχων σχεδιασμός ελέγχει
            μόνο αν τα PVs ενός μόνο Pod μπορούν να χωρέσουν σε έναν
            κόμβο, αλλά όχι τα PVs πολλαπλών Pods. Παρόλο που μια
            λανθασμένη απόφαση για την κλιμάκωση προς τα κάτω λόγω αυτής
            της έλλειψης θα διορθωθεί με μια επόμενη κλιμάκωση προς τα
            πάνω, η λήψη της σωστής απόφασης εξ' αρχής θα ήταν πολύ πιο
            αποτελεσματική.
      \item Να επεκτείνουμε την τρέχουσα υλοποίηση της διεπαφή
            \co{Estimator} του Autoscaler, δηλαδή τον
            \co{BinPackingEstimator} ώστε να λαμβάνει υπόψη την
            αποθηκευτική χωρητικότητα  και να υπολογίζει, έτσι, πόσοι
            κόμβοι απαιτούνται όσον αφορά τα αιτήματα αποθηκευτικού χώρου
            πολλαπλών Pods ενός StatefulSet. Ο τρέχων σχεδιασμός προσθέτει
            έναν έναν τους κόμβους μέχρι όλα τα Pods να λάβουν τον
            ζητούμενο αποθηκευτικό χώρο. Θα ήταν πολύ πιο αποδοτικό αν
            ξέραμε εξ' αρχής πόσοι κόμβοι χρειάζονται ώστε να προστεθούν
            μονομιάς στη συστοιχία.
\end{itemize}


Τέλος, όπως έχουμε ήδη αναφέρει, ο μακροπρόθεσμος στόχος μας είναι να
συνεισφέρουμε το σχεδιασμό μας στο upstream project του Kubernetes.