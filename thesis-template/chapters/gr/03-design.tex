%\chapter{Σχεδίαση} \label{chapter:gr-design}
\chapter{Σχεδίαση} \label{chapter:gr-design}

Σε αυτό το κεφάλαιο, περιγράφουμε το σχεδιασμό και τους αλγορίθμους που καθορίζουν τη
λειτουργία του Cluster Autoscaler και του Scheduler, επισημαίνουμε τις
υπάρχουσες ελλείψεις όσον αφορά την τοπική αποθήκευση δεδομένων και προτείνουμε σχεδιαστικές
βελτιώσεις για να καταστεί δυνατή η απρόσκοπτη χρονοδρομολόγηση και αυτόματη κλιμάκωση όταν
χρησιμοποιούνται τοπικοί μόνιμοι τόμοι.

\section{Σχεδιαστική Λογική και Στόχοι}

Στόχος μας είναι να επεκτείνουμε
τον Kubernetes Scheduler και τον Kubernetes Cluster Autoscaler έτσι ώστε να
λειτουργούν απρόσκοπτα με φορτία εργασίας που χρησιμοποιούν τόμους με τοπικό αποθηκευτικό χώρο.

Πιο συγκεκριμένα, σκοπεύουμε:
\begin{itemize}
    \tightlist
    \item Να επεκτείνουμε τον Kubernetes Scheduler ώστε να λαμβάνει υπόψη του την
    αποθηκευτική ικανότητα κάθε κόμβου και να δρομολογεί τα Pods σε κόμβους που έχουν
    επαρκή ελεύθερο αποθηκευτικό χώρο για να φιλοξενήσουν τους τόμους που ζητάει το κάθε Pod.
    \item Να επεκτείνουμε τον Cluster Autoscaler ώστε να εκτελεί κλιμάκωση προς τα κάτω των κόμβων που εκτελούν Pods
    που χρησιμοποιούν τοπικούς τόμους αποθήκευσης, εξασφαλίζοντας ότι δημιουργούνται αντίγραφα ασφαλείας των δεδομένων
    πριν από τη διαγραφή του κόμβου από τη συστοιχία, προκειμένου να ανακτηθούν αργότερα σε
    έναν διαφορετικό κόμβο.
    \item Να επεκτείνουμε τον Cluster Autoscaler για να ελέγχει αν υπάρχει
    αρκετός διαθέσιμος αποθηκευτικός χώρος σε άλλους κόμβους για τη φιλοξενία
    των τόμων ενός Pod που εκτελείται σε έναν κόμβο, όταν ο κόμβος αξιολογείται
    για μείωση της κλίμακας.
    \item Να επεκτείνουμε τον Cluster Autoscaler ώστε κατά την κλιμάκωση προς τα πάνω να λαμβάνει υπόψη την αποθηκευτική
    χωρητικότητα των κόμβων που προσθέτει σε μια συστοιχία και να επιλέγει τον κατάλληλο
    τύπο κόμβου που διαθέτει επαρκή τοπική χωρητικότητα αποθήκευσης.
\end{itemize}


\section{Ο Μηχανισμός Τοπικών Τόμων του Rok}

Τα τοπικά δεδομένα που ζουν σε έναν κόμβο χρειάζονται ένα μηχανισμό για τη
δημιουργία αντιγράφων ασφαλείας εάν ο κόμβος πρόκειται να αφαιρεθεί από τη
συστοιχία, διαφορετικά τα δεδομένα θα χαθούν και δεν θα είναι δυνατή η ανάκτησή
τους.

Το Rok υλοποιεί έναν μηχανισμό που επιτρέπει τη μετακίνηση τόμων μεταξύ των
κόμβων κόμβων μιας συστοιχίας. Αξιοποιεί ένα εξωτερικό σύστημα αποθήκευσης, όπως
το S3 της Amazon, για τη δημιουργία αντιγράφων ασφαλείας των τοπικών τόμων και
δύναται να τους ανακτήσει σε οποιονδήποτε άλλο κόμβο, εφόσον ζητηθεί. Το Rok
αποκαλεί τη διαδικασία μετακίνησης ενός τοπικού τόμου στο S3 ως
\textit{``unpinning''} του τόμου και τη διαδικασία επαναφοράς των δεδομένων ενός
τόμου από το S3 σε έναν κόμβο ως \textit{``pinning''} του τόμου. Περιγράφουμε
αυτόν τον μηχανισμό σε μεγαλύτερο βάθος στην ενότητα που ακολουθεί.

\subsection{Pinning και Unpinning των Τοπικών Τόμων του Rok}

\label{section:gr-rok-volume-pinning}

Όταν δημιουργείται ένας τοπικός τόμος σε έναν κόμβο, το αντίστοιχο
\texttt{PersistentVolume} αντικείμενο στον API Server που αναπαριστά τον τόμο
διαθέτει ένα πεδίο node affinity, το οποίο καθορίζει τους κόμβους από τους
οποίους είναι προσβάσιμος ο τόμος. Στην περίπτωση του τοπικού αποθηκευτικού
χώρου, το node affinity του τόμου ορίζεται έτσι ώστε να επιλέγει μόνο τον κόμβο
στον οποίο βρίσκονται τοπικά τα δεδομένα του τόμου.

Το Rok εισάγει την εξής ορολογία:
\begin{itemize}
      \tightlist
      \item \textit{Pinned PV}: Ένα PV που αντιπροσωπεύει έναν τόμο του οποίου
            τα δεδομένα ζουν τοπικά στον κόμβο. Αυτό το PV διαθέτει node
            affinity για να υποδεικνύει ότι είναι προσβάσιμο μόνο από τον
            συγκεκριμένο κόμβο.
      \item \textit{Unpinned PV}: Ένα PV που αντιπροσωπεύει έναν τόμο του οποίου
            τα δεδομένα βρίσκονταν προηγουμένως τοπικά, αλλά πλέον έχουν
            μεταφερθεί στο Amazon S3. Το node affinity του PV  είναι κενό για να
            υποδεικνύει ότι είναι προσβάσιμο από κάθε κόμβο της συστοιχίας. Ο
            χρονοδρομολογητής θα θεωρήσει αυτό το PV ως προσβάσιμο από κάθε
            κόμβο και, συνεπώς, δεν θα περιορίσει τη χρονοδρομολόγηση του
            αντίστοιχου Pod στον κόμβο όπου τα δεδομένα ζούσαν προηγουμένως.
\end{itemize}

Ένα pinned PV μπορεί να γίνει unpinned με τη διαδικασία του \textit{unpinning}.
Ένα unpinned PV μπορεί να γίνει pinned με τη διαδικασία του \textit{pinning}.
Ένας τόμος μπορεί να αλλάξει από pinned σε unpinned και αντίστροφα πολλές φορές,
επιτρέποντας ουσιαστικά στον τόμο να μετακινηθεί σε διαφορετικούς κόμβους της
συστοιχίας όσες φορές χρειάζεται.


Το Rok, και συγκεκριμένα ο ελεγκτής Rok CSI, υλοποιεί τον ακόλουθο μηχανισμό:
\begin{enumerate}
      \tightlist
      \item Παρακολουθεί τη συστοιχία για να βρει κόμβους που είναι
            unschedulable.
      \item Βρίσκει τους τόμους σε κάθε μη χρονοδρομολογήσιμο κόμβο που δεν
            χρησιμοποιούνται από κανένα Pod.
      \item Εκκινεί τη διαδικασία unpinning των τόμων αυτών: δημιουργεί με
            αποδοτικό τρόπο snapshots των δεδομένων του τόμου στο Amazon S3.
      \item Αφαιρεί το node affinity από το PV. Ας σημειωθεί ότι το πεδίο
            \co{nodeAffinity} ενός PV είναι αμετάβλητο και για να ξεπεραστεί
            αυτός ο περιορισμός, το Rok διαγράφει το PV και το αναδημιουργεί
            στιγμιαία.
\end{enumerate}

Το Rok υλοποιεί τον ακόλουθο μηχανισμό για το pinning ενός PV:
\begin{enumerate}
      \tightlist
      \item Ο χρονοδρομολογητής χρονοδρομολογεί ένα Pod που αναφέρεται στο
            unpinned PV (μέσω του PVC του).
      \item Ο ελεγκτής \co{attachDetach} του Kubernetes δημιουργεί ένα
            αντικείμενο\\ \co{VolumeAttachment} για να σηματοδοτήσει στον
            external attacher να προσαρτήσει τον τόμο στον κόμβο.
      \item Ο external attacher βλέπει το VolumeAttachment και στέλνει μία κλήση
            \EN{ControllerPublishVolume} στον ελεγκτή Rok CSI.
      \item O ελεγκτής του Rok CSI  δημιουργεί έναν λογικό τόμο στον Rok VG και
            επαναφέρει τα δεδομένα από το Amazon S3 στον λογικό τόμο.
      \item Ο ελεγκτής του Rok CSI θέτει το κατάλληλο node affinity στο PV ώστε
            να υποδεικνύει ότι είναι προσβάσιμο μόνο από τον κόμβο στον οποίο
            έγινε η ανάκτηση του τόμου.
\end{enumerate}

\subsection{Ο Μηχανισμός Προστασίας Τοπικών Τόμων του Rok}
\label{section:background-gr-rok-csi-guard}


Τα εργαλεία συντήρησης και αναβάθμισης του Kubernetes βασίζονται στη διαδικασία
drain. Ουσιαστικά, πριν από οποιαδήποτε ενέργεια για την αφαίρεση ή αναβάθμιση
ενός κόμβου στη συστοιχία, τα εργαλεία κάνουν drain τον κόμβο (\co{kubectl
      drain}), ώστε να επισημανθεί ως unschedulable και να εκδιώξουν με ασφάλεια όλα
τα Pods του κόμβου. Ο ίδιος ο Cluster Autoscaler, χρησιμοποιεί επίσης τη
λειτουργία drain πριν από την αφαίρεση ενός κόμβου.

Το Rok διαθέτει έναν μηχανισμό προκειμένου να διευκολύνει τις αναβαθμίσεις μιας
συστοιχίας και να διασφαλίσει ότι οι κόμβοι δεν αφαιρούνται πριν από τη
δημιουργία στιγμιότυπων όλων των τοπικών τόμων.

Ο μηχανισμός αξιοποιεί Pods με κατάλληλα PodDisruptionBudgets για να μπλοκάρει
την εκδίωξή τους. Όσο αποτυγχάνει το eviction ενός Pod, η λειτουργία drain
αποτυγχάνει. Ο μηχανισμός λειτουργεί ως εξής:

\begin{enumerate}
      \tightlist
      \item Ο Rok Operator δημιουργεί ένα αντικείμενο \co{Deployment}
            \textit{για κάθε κόμβο} στη συστοιχία. To Deployment του κάθε κόμβου
            δημιουργεί \textit{ακριβώς ένα} replica Pod, το οποίο διαθέτει node
            affinity που ταιριάζει μόνο στον συγκεκριμένο κόμβο. Το Rok ονομάζει
            αυτά τα Pods ``\textit{CSI Guard Pod}'', καθώς φυλάσσουν τον
            αποθηκευτικό χώρο του κόμβου.
      \item Ο Rok Operator δημιουργεί ένα αντικείμενο \co{PodDisruptionBudget}
            για κάθε Rok CSI Guard Deployment. Το PodDisruptionBudget απαιτεί σε
            κάθε στιγμή να υπάρχει τουλάχιστον ένα Rok CSI Guard Pod του
            Deployment. Με αυτό τον τρόπο, όσο υπάρχει το PodDisruptionBudget,
            το eviction του αντίστοιχου Rok CSI Guar Pod θα αποτυγχάνει.
      \item Η διαδικασία drain χαρακτηρίζει τον κόμβο ως unschedulable και
            ξεκινά την εκδίωξή των Pods του κόμβου. Η εκδίωξη του CSI Guard
            αποτυγχάνει λόγω του PodDisruptionBudget.
      \item Ο Rok Operator, ελέγχει αν όλοι οι τοπικοί τόμοι Rok στον
            unschedulable  κόμβο έχουν γίνει unpinned.  Αν ισχύει αυτή η
            συνθήκη, αφαιρεί το PodDisruptionBudget που αντιστοιχεί στο CSI
            Guard του κόμβου.
      \item Εφόσον το PDB αφαιρέθηκε, επιτυγχάνει η εκδίωξή του Rok CSI Guard
            Pod του κόμβου και η διαδικασία drain ολοκληρώνεται επιτυχώς.
\end{enumerate}

\section{The Kubernetes Scheduler} \label{section:background_scheduler}

A scheduler on a Kubernetes cluster is a component that watches for newly
created Pods that have no node assigned. For every Pod the scheduler discovers,
the scheduler becomes responsible for finding the best node for that Pod to run
on. In this section, we will present the background needed to understand how
scheduling occurs.

\subsection{Scheduling Fundamentals} \label{section:scheduling-fundamentals}

\paragraph*{Multiple schedulers}

A \textit{scheduler} is a component on a Kubernetes cluster that assigns a node
for each Pod to run on. Kubernetes clusters ship with the default scheduler,
namely \co{kube-scheduler}, which runs as part of the control plane. However,
multiple schedulers may run on a cluster; in this case, each Pod must specify
the scheduler's name that shall handle it by setting its \co{spec.schedulerName}
field to the name of the preferred scheduler. If a scheduler name is not
explicitly specified, the Pod will be scheduled using the default scheduler.

\lstinputlisting[language=yaml,caption={A Pod to be scheduled by \co{another-scheduler}}]{code/pod-scheduler.yaml}

\paragraph*{Feasible nodes}
Each Pod has different requirements, e.g., CPU, memory, and node affinity, which
affect which nodes the Pod can run on. Factors that need to be taken into
account for scheduling decisions include individual and collective resource
requirements, hardware / software / policy constraints, affinity and
anti-affinity specifications, data locality, and inter-workload interference.
The nodes that meet the scheduling requirements for a Pod are called
\textit{feasible} nodes. If none of the nodes in the cluster are feasible, the
Pod will remain unscheduled, i.e., it will not be assigned any node to run on.

\paragraph*{The \co{kube-scheduler}} The Kubernetes default scheduler,
\co{kube-scheduler} selects a node for the Pod in a 3-step operation:
\begin{enumerate}
      \tightlist
      \item \textit{Filtering}: The scheduler finds the set of nodes where it is
            feasible to schedule the Pod. It executes a series of filtering
            plugins that evaluate whether the Pod can be placed on the examined
            node. For example, the \co{PodFitsResources} filter checks whether a
            candidate node has enough resources (CPU, RAM, GPU) to meet the
            Pod's specific resource requests. A node is \textit{feasible} if all
            the filter plugins consider the node as feasible for the Pod. This
            step calculates a node list with suitable nodes; if the list is
            empty, the Pod is \textit{unschedulable}.
      \item \textit{Scoring}: The scheduler ranks the feasible nodes to choose
            the most suitable Pod placement. The scheduler assigns a score to
            each feasible node based on the active scoring rules.
      \item The scheduler assigns the Pod to the node with the highest ranking.
            If there is more than one node with equal scores, it randomly
            selects one of these. The scheduler assigns the Pod to a node by
            setting the \co{spec.nodeName} field of the Pod to the name of the
            selected node.
\end{enumerate}


\subsection{The Scheduling
      Framework}\label{section:background_scheduling_framework}

The scheduling framework is a pluggable architecture for the Kubernetes
scheduler. It adds a set of \textit{plugin} APIs to the existing scheduler.
Plugins are compiled into the scheduler. The APIs allow most scheduling features
to be implemented as plugins while keeping the scheduling core lightweight and
maintainable.

\paragraph*{Scheduling \& binding cycles}

The scheduler stores a queue of Pods waiting to be scheduled. It picks a Pod
from the queue and attempts to schedule it. Each attempt to schedule a pod is
split into two phases:

\begin{itemize}
      \tightlist
      \item \textit{Scheduling cycle}: Selects a node for the Pod. The
            scheduling cycles of different Pods are run \textit{serially}.
      \item \textit{Binding cycle}: Applies that decision for the Pod to the
            cluster. Multiple binding cycles for different Pods are run
            \textit{concurrently}.
\end{itemize}

A scheduling cycle and binding cycle together are referred to as the
``\textit{scheduling context}''. A scheduling cycle or binding cycle can be
aborted if the Pod is determined to be unschedulable or if there is an internal
error. The Pod will be returned to the queue and retried. If a binding cycle is
aborted, it will trigger the \texttt{Unreserve} method in the Reserve plugin.


The scheduling framework exposes some extension points. The plugins are
registered to be called at one or more of these extension points. One plugin may
register at multiple extension points. The extension points are illustrated in
Figure~\ref{fig:scheduling-plugins}.

\begin{figure}[ht]
      \centering
      \includegraphics[width=\textwidth]{chapters/background/img/scheduler.png}
      \caption{The extension points of the scheduling framework}
      \label{fig:scheduling-plugins}
\end{figure}


\paragraph*{Scheduler framework extension points}

The scheduler framework offers the following extensions points where each plugin
can be registered:

\begin{itemize}
      \item
            \textbf{\co{Queue sort}}: The scheduler uses these plugins to sort
            Pods in the scheduling queue. A queue sort plugin essentially will
            provide a \co{less(pod1, Pod2)} function. Only one queue sort plugin
            may be enabled at a time.
      \item
            \textbf{\co{PreFilter}}: The scheduler uses these plugins to
            pre-process info about the Pod or to check certain conditions that
            the cluster or the Pod must meet. A PreFilter plugin should
            implement a \co{PreFilter} function. If \co{PreFilter} returns an
            error, the scheduler will abort the scheduling cycle..
      \item
            \textbf{\co{Filter}}: The scheduler uses these plugins to
            \textit{filter out} nodes that cannot run the Pod. For each node,
            the scheduler will call the filter plugins in their configured
            order. If any filter plugin marks the node as \textit{infeasible},
            the scheduler will not call the remaining plugins for that evaluated
            node. The scheduler may evaluate nodes concurrently, and thus, it
            may call a \co{Filter} plugin more than once in the same scheduling
            cycle.
      \item
            \textbf{\co{PostFilter}}: The scheduler calls these plugins in their
            configured order after the \co{Filter} phase, but only if it did not
            find any feasible nodes for the Pod. If any of them marks the node
            as schedulable, the scheduler will not call the remaining plugins. A
            typical PostFilter implementation is the \co{preemption} plugin,
            which tries to make the Pod schedulable by preempting other Pods.
      \item
            \textbf{\co{PreScore}}: This is an informational extension point for
            performing pre-scoring work. The scheduler will call the plugins
            with a list of nodes that passed the filtering phase. A plugin may
            use this data to update the internal state or to generate logs or
            metrics.
      \item
            \textbf{\co{Scoring}}: This extension point has two phases:
            \begin{itemize}
                  \item
                        The first phase is called ``\textit{score}'' and is used
                        to rank nodes that have passed the filtering phase. The
                        scheduler will call the \texttt{Score} method of each
                        scoring plugin for each node.
                  \item The second phase is ``\textit{normalize scoring}'' and
                        is used to modify scores before the scheduler computes a
                        final ranking of nodes.
            \end{itemize}

      \item
            \textbf{\co{Reserve}}: A plugin that implements the Reserve
            extension has two methods, namely \texttt{Reserve} and
            \texttt{Unreserve}. Plugins that maintain runtime state, i.e.,
            \textit{stateful plugins}, should use these phases to reserve and
            unreserve any resources in the internal state of the scheduler.

            The \emph{Reserve} phase exists to prevent race conditions while the
            scheduler waits for the bind to succeed. The scheduler executes it
            before it binds a Pod to its designated node. The \texttt{Reserve}
            method of each Reserve plugin may succeed or fail.

            If the \texttt{Reserve} method of all plugins succeeds, the
            scheduler considers the Reserve phase to be successful and executes
            the rest of the scheduling cycle and the binding cycle.

            If one \texttt{Reserve} method call fails, the scheduler will not
            execute the subsequent phases and will run the \co{Unreserve} phase
            instead. The \emph{Unreserve} phase exists to clean up the state
            associated with the reserved Pod. The scheduler calls the
            \co{Unreserve} method of all the \co{Reserve} plugins in the reverse
            order of \co{Reserve} method calls.
      \item
            \textbf{\co{Permit}}: The scheduler uses these plugins to prevent or
            delay the binding of a Pod. It executes them as the last step of a
            scheduling cycle; however, it waits for the permit phase to execute
            successfully at the beginning of a binding cycle, before executing
            the \texttt{PreBind} plugins.
\end{itemize}


\paragraph*{The \co{CycleState} struct}
\label{section:cycle-state}

The various plugins running in the scheduling context of a single Pod share a
common \co{CycleState} struct. \co{CycleState} provides a mechanism for plugins
to store and retrieve arbitrary data. Data stored in the \co{CycleState} by one
plugin can be read, altered, or deleted by another. \co{CycleState} does not
provide any data protection, as all plugins are assumed to be trusted.


\subsection{The VolumeBinding Plugin}\label{section:background_volume_binding}

The \texttt{VolumeBinding} plugin of the Kubernetes Scheduler binds Pod volumes
in scheduling. The VolumeBinding plugin is registered on the \texttt{PreFilter},
\texttt{Filter}, \texttt{PreBind}, \texttt{Reserve}, and \texttt{Unreserve}
extension points. The \texttt{PreFilter}, \texttt{Filter}, and \texttt{PreBind}
phases of the plugin are of particular interest in the context of this thesis.
We explain here briefly the operations that take place:

\begin{itemize}
      \tightlist
      \item \co{PreFilter}: Checks if a Pod has all its immediate (PVCs that
            request a storage class with \co{Immediate} binding mode) PVCs
            bound. If not all immediate PVCs are bound, the plugin returns an
            \co{UnschedulableAndUnresolvable} error.
      \item \co{Filter}:  Evaluates if a Pod fits a node due to the volumes it
            requests:
            \begin{itemize}
                  \tightlist
                  \item For \textit{bound PVCs}, it checks that the
                        corresponding node affinity of the PV of each PVC is
                        satisfied by the given node.
                  \item For each \textit{unbound PVC}, it tries to find an
                        available PVs that satisfies the PVC requirements
                        (access mode, requested capacity) and has node affinity
                        that matches the given node.
                  \item For each \co{unbound PVC} that did not find a matching
                        PV (hereafter referred to as the ``PVCs to provision''),
                        it checks if there is enough space on the node to
                        provision a volume.


                  \item The \co{Filter} method returns true if the following
                        conditions hold:
                        \begin{enumerate}
                              \item  he PVs the (bound) PVCs are bound to are
                                    accessible from the node
                              \item the unbound PVCs can be matched to an
                                    existing available PV that is accessible
                                    from the node or  have the storage driver
                                    dynamically provision such a PV, if there is
                                    enough storage.
                        \end{enumerate}

            \end{itemize}
      \item \co{PreBind}: PreBind updates the API Server with the assumed
            bindings and waits until the PersistentVolume controller has wholly
            finished the binding operation. If binding errors, times out or gets
            undone, then an error will be returned to retry scheduling.
\end{itemize}

\section{The Kubernetes Cluster Autoscaler}\label{section:background_autoscaler}

There are three different types of autoscaling in Kubernetes:
\begin{itemize}
  \tightlist
  \item \textbf{Cluster Autoscaler (Autoscaler)}: adjusts the number of nodes in the
  cluster when Pods fail to schedule or when nodes are underutilized.
  \item \textbf{Horizontal Pod Autoscaler (HPA)}: adjusts the number of replicas
  of an application.
  \item \textbf{Vertical Pod Autoscaler (VPA)}: adjusts the resource requests and limits of the containers of a Pod.
\end{itemize}
\begin{figure}[ht]
  \centering
  \includegraphics[width=0.9\textwidth]{resources/hpa-autoscaling-blue.png}
  \caption{Kubernetes Horizontal Pod Autoscaling}
\end{figure}

\begin{figure}[ht]
  \centering
  \includegraphics[width=0.9\textwidth]{resources/vpa-diagram-blue.png}
  \caption{Kubernetes Vertical Pod Autoscaling}
\end{figure}

\begin{figure}[ht]
  \centering
  \includegraphics[width=0.9\textwidth]{resources/ca-process-blue.png}
  \caption{Kubernetes Cluster Autoscaling}
\end{figure}

%TODO: COLORBOX
\textbf{In the context of this thesis, we only discuss the Cluster Autoscaler. We will
refer to it as the ``Autoscaler'', and we will imply that we talk about the
Cluster Autoscaler.
}
\subsection{Cluster Autoscaling Fundamentals}

The Cluster Autoscaler is a component that automatically adjusts the Kubernetes
cluster size. It automatically adds or removes nodes based on the Pods' resource
requests compared to the nodes' available resources (see
section~\ref{section:pod-requests}); it does not measure the live CPU and memory
usage.

If the scheduler could not schedule any Pods, the Autoscaler will add new nodes
in the cluster to help the Pods get scheduled. This action is called
\textit{scale-up} of the cluster (or, equivalently, \textit{scale-out}). If any
nodes in the cluster are not needed, the Autoscaler will remove these nodes.
This action is called \textit{scale-down} of the cluster (or, equivalently,
\textit{scale-in}).

The following paragraphs explain the basic principles of scaling up and down a
cluster.

\subsection{Scale-Up Procedure}

\label{section:backgroud-scale-up}

The Autoscaler periodically checks for any unschedulable Pods on the cluster and
tries to find a new place for them to run.

The Autoscaler assumes that the underlying cluster runs on top of some
\textit{node groups}. Inside a node group, all machines have the same capacity
and assigned labels. Increasing the size of a node group will create a new node
that will be similar to those already in the cluster.

Based on the above assumption, the Autoscaler creates template nodes for each
node group and checks if any unschedulable Pod would fit on a new node. If there
are multiple node groups that, if increased, would help some Pods to run, the
administrator can specify different strategies for the Autoscaler to choose
which node group it shall increase. This procedure may require multiple
iterations before all the Pods are eventually scheduled.


\subsection{Scale-Down Procedure}
\label{section:backgroud-scale-down}

The Autoscaler periodically checks which nodes are unneeded, provided that no
scale-up is needed. A node is considered for removal if all the following
conditions hold:

\begin{itemize}
  \tightlist
  \item The sum of CPU and memory requests of all Pods running on this node is
  less than 50\% of the node's allocatable resources. This threshold is
  configurable.
  \item All Pods running on the node (except Pods created by Daemons) can be
  moved to any other node in the cluster. 
\end{itemize}

The Autoscaler removes a node if it remains unneeded for more than 10 minutes
(configurable duration). It terminates one \textit{non-empty} node at a time to
reduce the risk of creating new unschedulable Pods. On the other hand, it
terminates \textit{Empty} nodes (nodes that run only DaemonSet Pods) in bulk.
When a non-empty node is terminated, as mentioned above, all Pods should be
migrated elsewhere. It achieves this by marking the node unschedulable and then
evicting the Pods that run on it. As soon as each Pod is successfully evicted,
the scheduler shall schedule it on a different node.
