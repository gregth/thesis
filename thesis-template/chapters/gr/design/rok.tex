\section{Ο Μηχανισμός Τοπικών Τόμων του Rok}

Τα τοπικά δεδομένα που ζουν σε έναν κόμβο χρειάζονται ένα μηχανισμό για τη
δημιουργία αντιγράφων ασφαλείας εάν ο κόμβος πρόκειται να αφαιρεθεί από τη
συστοιχία, διαφορετικά τα δεδομένα θα χαθούν και δεν θα είναι δυνατή η ανάκτησή
τους.

Το Rok υλοποιεί έναν μηχανισμό που επιτρέπει τη μετακίνηση τόμων μεταξύ των
κόμβων κόμβων μιας συστοιχίας. Αξιοποιεί ένα εξωτερικό σύστημα αποθήκευσης, όπως
το S3 της Amazon, για τη δημιουργία αντιγράφων ασφαλείας των τοπικών τόμων και
δύναται να τους ανακτήσει σε οποιονδήποτε άλλο κόμβο, εφόσον ζητηθεί. Το Rok
αποκαλεί τη διαδικασία μετακίνησης ενός τοπικού τόμου στο S3 ως
\textit{``unpinning''} του τόμου και τη διαδικασία επαναφοράς των δεδομένων ενός
τόμου από το S3 σε έναν κόμβο ως \textit{``pinning''} του τόμου. Περιγράφουμε
αυτόν τον μηχανισμό σε μεγαλύτερο βάθος στην ενότητα που ακολουθεί.

\subsection{Pinning και Unpinning των Τοπικών Τόμων του Rok}

\label{section:gr-rok-volume-pinning}

Όταν δημιουργείται ένας τοπικός τόμος σε έναν κόμβο, το αντίστοιχο
\texttt{PersistentVolume} αντικείμενο στον API Server που αναπαριστά τον τόμο
διαθέτει ένα πεδίο node affinity, το οποίο καθορίζει τους κόμβους από τους
οποίους είναι προσβάσιμος ο τόμος. Στην περίπτωση του τοπικού αποθηκευτικού
χώρου, το node affinity του τόμου ορίζεται έτσι ώστε να επιλέγει μόνο τον κόμβο
στον οποίο βρίσκονται τοπικά τα δεδομένα του τόμου.

Το Rok εισάγει την εξής ορολογία:
\begin{itemize}
      \tightlist
      \item \textit{Pinned PV}: Ένα PV που αντιπροσωπεύει έναν τόμο του οποίου
            τα δεδομένα ζουν τοπικά στον κόμβο. Αυτό το PV διαθέτει node
            affinity για να υποδεικνύει ότι είναι προσβάσιμο μόνο από τον
            συγκεκριμένο κόμβο.
      \item \textit{Unpinned PV}: Ένα PV που αντιπροσωπεύει έναν τόμο του οποίου
            τα δεδομένα βρίσκονταν προηγουμένως τοπικά, αλλά πλέον έχουν
            μεταφερθεί στο Amazon S3. Το node affinity του PV  είναι κενό για να
            υποδεικνύει ότι είναι προσβάσιμο από κάθε κόμβο της συστοιχίας. Ο
            χρονοδρομολογητής θα θεωρήσει αυτό το PV ως προσβάσιμο από κάθε
            κόμβο και, συνεπώς, δεν θα περιορίσει τη χρονοδρομολόγηση του
            αντίστοιχου Pod στον κόμβο όπου τα δεδομένα ζούσαν προηγουμένως.
\end{itemize}

Ένα pinned PV μπορεί να γίνει unpinned με τη διαδικασία του \textit{unpinning}.
Ένα unpinned PV μπορεί να γίνει pinned με τη διαδικασία του \textit{pinning}.
Ένας τόμος μπορεί να αλλάξει από pinned σε unpinned και αντίστροφα πολλές φορές,
επιτρέποντας ουσιαστικά στον τόμο να μετακινηθεί σε διαφορετικούς κόμβους της
συστοιχίας όσες φορές χρειάζεται.


Το Rok, και συγκεκριμένα ο ελεγκτής Rok CSI, υλοποιεί τον ακόλουθο μηχανισμό:
\begin{enumerate}
      \tightlist
      \item Παρακολουθεί τη συστοιχία για να βρει κόμβους που είναι
            unschedulable.
      \item Βρίσκει τους τόμους σε κάθε μη χρονοδρομολογήσιμο κόμβο που δεν
            χρησιμοποιούνται από κανένα Pod.
      \item Εκκινεί τη διαδικασία unpinning των τόμων αυτών: δημιουργεί με
            αποδοτικό τρόπο snapshots των δεδομένων του τόμου στο Amazon S3.
      \item Αφαιρεί το node affinity από το PV. Ας σημειωθεί ότι το πεδίο
            \co{nodeAffinity} ενός PV είναι αμετάβλητο και για να ξεπεραστεί
            αυτός ο περιορισμός, το Rok διαγράφει το PV και το αναδημιουργεί
            στιγμιαία.
\end{enumerate}

Το Rok υλοποιεί τον ακόλουθο μηχανισμό για το pinning ενός PV:
\begin{enumerate}
      \tightlist
      \item Ο χρονοδρομολογητής χρονοδρομολογεί ένα Pod που αναφέρεται στο
            unpinned PV (μέσω του PVC του).
      \item Ο ελεγκτής \co{attachDetach} του Kubernetes δημιουργεί ένα
            αντικείμενο\\ \co{VolumeAttachment} για να σηματοδοτήσει στον
            external attacher να προσαρτήσει τον τόμο στον κόμβο.
      \item Ο external attacher βλέπει το VolumeAttachment και στέλνει μία κλήση
            \EN{ControllerPublishVolume} στον ελεγκτή Rok CSI.
      \item O ελεγκτής του Rok CSI  δημιουργεί έναν λογικό τόμο στον Rok VG και
            επαναφέρει τα δεδομένα από το Amazon S3 στον λογικό τόμο.
      \item Ο ελεγκτής του Rok CSI θέτει το κατάλληλο node affinity στο PV ώστε
            να υποδεικνύει ότι είναι προσβάσιμο μόνο από τον κόμβο στον οποίο
            έγινε η ανάκτηση του τόμου.
\end{enumerate}

\subsection{Ο Μηχανισμός Προστασίας Τοπικών Τόμων του Rok}
\label{section:background-gr-rok-csi-guard}


Τα εργαλεία συντήρησης και αναβάθμισης του Kubernetes βασίζονται στη διαδικασία
drain. Ουσιαστικά, πριν από οποιαδήποτε ενέργεια για την αφαίρεση ή αναβάθμιση
ενός κόμβου στη συστοιχία, τα εργαλεία κάνουν drain τον κόμβο (\co{kubectl
      drain}), ώστε να επισημανθεί ως unschedulable και να εκδιώξουν με ασφάλεια όλα
τα Pods του κόμβου. Ο ίδιος ο Cluster Autoscaler, χρησιμοποιεί επίσης τη
λειτουργία drain πριν από την αφαίρεση ενός κόμβου.

Το Rok διαθέτει έναν μηχανισμό προκειμένου να διευκολύνει τις αναβαθμίσεις μιας
συστοιχίας και να διασφαλίσει ότι οι κόμβοι δεν αφαιρούνται πριν από τη
δημιουργία στιγμιότυπων όλων των τοπικών τόμων.

Ο μηχανισμός αξιοποιεί Pods με κατάλληλα PodDisruptionBudgets για να μπλοκάρει
την εκδίωξή τους. Όσο αποτυγχάνει το eviction ενός Pod, η λειτουργία drain
αποτυγχάνει. Ο μηχανισμός λειτουργεί ως εξής:

\begin{enumerate}
      \tightlist
      \item Ο Rok Operator δημιουργεί ένα αντικείμενο \co{Deployment}
            \textit{για κάθε κόμβο} στη συστοιχία. To Deployment του κάθε κόμβου
            δημιουργεί \textit{ακριβώς ένα} replica Pod, το οποίο διαθέτει node
            affinity που ταιριάζει μόνο στον συγκεκριμένο κόμβο. Το Rok ονομάζει
            αυτά τα Pods ``\textit{CSI Guard Pod}'', καθώς φυλάσσουν τον
            αποθηκευτικό χώρο του κόμβου.
      \item Ο Rok Operator δημιουργεί ένα αντικείμενο \co{PodDisruptionBudget}
            για κάθε Rok CSI Guard Deployment. Το PodDisruptionBudget απαιτεί σε
            κάθε στιγμή να υπάρχει τουλάχιστον ένα Rok CSI Guard Pod του
            Deployment. Με αυτό τον τρόπο, όσο υπάρχει το PodDisruptionBudget,
            το eviction του αντίστοιχου Rok CSI Guar Pod θα αποτυγχάνει.
      \item Η διαδικασία drain χαρακτηρίζει τον κόμβο ως unschedulable και
            ξεκινά την εκδίωξή των Pods του κόμβου. Η εκδίωξη του CSI Guard
            αποτυγχάνει λόγω του PodDisruptionBudget.
      \item Ο Rok Operator, ελέγχει αν όλοι οι τοπικοί τόμοι Rok στον
            unschedulable  κόμβο έχουν γίνει unpinned.  Αν ισχύει αυτή η
            συνθήκη, αφαιρεί το PodDisruptionBudget που αντιστοιχεί στο CSI
            Guard του κόμβου.
      \item Εφόσον το PDB αφαιρέθηκε, επιτυγχάνει η εκδίωξή του Rok CSI Guard
            Pod του κόμβου και η διαδικασία drain ολοκληρώνεται επιτυχώς.
\end{enumerate}
