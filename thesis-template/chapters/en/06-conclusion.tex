\chapter{Conclusion} \label{chapter:conclusion}

Our journey has finally reached its end. In this chapter, we will restate our
contributions and summarize what our mechanism offers. Finally, we will close
this thesis by mentioning future work that can be done to enrich our mechanism
and bring it to its full potential.

\section{Concluding Remarks} \label{section:conclusion_concluding_remarks}

All in all, the primary goal of this thesis was to implement a design that would
enable seamless cluster autoscaling and scheduling with local persistent
storage. Not only did we achieve this goal, but our implementation was
successfully deployed to large production clusters of enterprises that requested
the feature.

Although the design we implemented is coupled with the Rok software --since it
provides an efficient mechanism for migrating local volumes around a cluster--,
the concepts and the design can be generalized to work with any other local
storage system. Our long-term goal, which extends beyond the context of this
thesis, is to generalize the design and push it upstream. That is a process that
we started to be involved in; we attend the meetings of the Kubernetes Storage
\footnote{https://github.com/kubernetes/community/blob/master/sig-storage/README.md}
and Autoscaling
\footnote{https://github.com/kubernetes/community/blob/master/sig-autoscaling/README.md}
Special Interest Groups, interacted with them on GitHub and plan to contribute
the whole design upstream actively. At the moment this text is written, we have
a few first Pull Requests merged
\footnote{https://github.com/kubernetes/autoscaler/pull/4877}
\footnote{https://github.com/kubernetes/autoscaler/pull/4842}.

\section{Future Work} \label{section:conclusion_future_work}

So far, we have implemented various enhancements for the Kubernetes Scheduler
and the Cluster Autoscaler, but there is always room for improvement. Since this
is an iterative process, in next iterations, we would like to offer these
enhancements:

\begin{itemize}
      \item Extend the Scheduler to reserve the storage (in the \co{Reserve}
            phase of the scheduling cycle) when scheduling a Pod to prevent race
            conditions.
      \item Extend the Cluster Autoscaler to consider the storage needed for the
            PVCs of multiple Pods when scaling down. The current design only
            checks if a single Pod's PVs can fit a node, but not if the PVs of
            multiple Pods fit a node. Although a wrong decision to scale down
            will be reverted by a subsequent scale-up, taking the decision would
            be much more effective.
      \item Extend the current implementation of the \co{Estimator} interface,
            i.e., the \co{BinPackingEstimator}, to consider the storage and
            calculate how many nodes are needed for the storage requests of
            multiple Pods of a StatefulSet. The current design adds nodes one by
            one till all the Pods get the requested storage. It would be much
            more efficient to know the number of nodes needed beforehand and add
            them to the cluster all at once.
\end{itemize}

Finally, as we already mentioned, our high-priority goal is to merge this work
upstream.