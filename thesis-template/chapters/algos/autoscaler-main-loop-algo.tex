\begin{algorithm}[H]
  \caption{Cluster Autoscaler: The main loop - RunOnce() method}\label{alg:autoscaler-main-loop}
  \begin{enumerate}[leftmargin=0cm]
    \tightlist
    \item
          \texttt{unschedulablePods} \lar Select Pods that do not have \texttt{spec.nodeName} set.
    \item
          \texttt{scheduledPods} \lar Select Pods that have \co{spec.nodeName} set.
    \item
          \texttt{allNodes} \lar List all the nodes of the cluster, by calling \co{ObtainNodesList()}.
    \item
          \texttt{readyNodes} \lar List the Ready nodes of the cluster, by calling \co{ObtainNodesList()}.

    \item
          \co{nodeGroups} \lar List the registered node groups of the cluster from the cloud provider.
    \item
          Take a snapshot of the cluster.
    \item
          For every node group in \co{nodeGroups}, generate its template node.
    \item For each node group in \co{nodeGroups} calculate the number of
          upcoming nodes (nodes that the Autoscaler has asked to be added but
          are not yet in the cluster) and add the same number of the node
          group's template nodes in the cluster snapshot.
    \item
          Run a scheduling simulation with the current cluster snapshot to determine if any of the \texttt{unschedulablePods} can be scheduled on the upcoming nodes.
    \item \lIf{any Pod in \co{unschedulablePods} is considered as schedulable in
            the simulation}{disable the scale-down for the current loop}
    \item \co{unschedulablePodsToHelp} \lar Pods from \co{unschedulablePods} that remained unschedulable in the simulation.
    \item \lIf{\co{unschedulablePodsToHelp} is empty}{do not scale-up}
    \item
          Else, try to scale-up, by calling \co{ScaleUp()}.
    \item
          \lIf{no scale-up was attempted}{proceed with the scale-down evaluation}
  \end{enumerate}
\end{algorithm}