\subsection{Kubernetes CSI: An End-to-End Story}

In this section, we aim to combine all the information by describing an
end-to-end story for the CSI.  We present the timeline of actions that take
place under the hood in order to provision a  volume dynamically.

The timeline of actions is the following:

\begin{enumerate}
	\tightlist
	\item The cluster administrator creates a StorageClass (in our case, the
	      \co{Rok} storage class) that specifies the CSI plug-in name
	      (\co{provisioner:rok.arrikto.com}):
	      \lstinputlisting[language=yaml]{code/rok-sc.yaml}

	\item A user creates a PersistentVolumeClaim that requests a volume of at
	      least 10 Gi with access mode \co{ReadWriteOnce} from the \co{Rok}
	      storage class.

	      \lstinputlisting[language=yaml]{code/pvc-rok.yaml}

	      Since the \co{Rok} StorageClass has \co{volumeBindingMode:
		      WaitForFirstConsumer}, the volume for the PVC will not be
	      provisioned as long as a Pod requesting the PVC is not scheduled
	      on a node. he PVC to be provisioned.
	\item  A user creates a Pod that uses the PVC:
	      \lstinputlisting[language=yaml]{code/pod-pvc-rok.yaml}
	\item The VolumeBinding plugin of the scheduler does not find any PV to
	      match the PVC. It signals the driver to provision the volume
	      dynamically: it annotates the PVC with the selected node annotation..
	\item The external provisioner sidecar that runs along with the Rok CSI
	      driver sees the annotation on the PVC and issues a \co{CreateVolume}
	      call against the Rok CSI Controller service to provision the volume.
	\item The Rok CSI controller provisions the volume and returns a successful
	      response to the external provisioner.
	\item The external provisioner creates a \co{PersistentVolume} object on the API
	      Server and binds it (the PV) to the  PVC.
	\item The PersistentVolume controller completes the bidirectional binding of
	      the PV and the PVC (by binding the PVC to the PV).
\end{enumerate}